\documentclass[a4paper,11pt]{article}
\usepackage[utf8]{inputenc}
\usepackage{graphicx}
\usepackage[english]{babel}
\usepackage[vmargin=3.5cm, top=2cm]{geometry}
\usepackage[linktocpage=true]{hyperref}
\usepackage{enumitem}
\usepackage{longtable}
\usepackage{pdfpages}
\usepackage{float}
\usepackage{hyperref}
\usepackage[section]{placeins}
\hypersetup{
   colorlinks,
   citecolor=black,
   filecolor=black,
   linkcolor=black,
   urlcolor=black
}

\begin{document}

\begin{titlepage}

\centering \parindent=0pt
\newcommand{\HRule}{\rule{\textwidth}{1mm}}
\vspace*{\stretch{1}} \HRule\\[1cm]\Huge\bfseries
Data Mining Report\\[0.7cm]
\HRule\\[4cm]  
\large by 
\\ Mikkel Stolborg
\\ Hlynur Örn Haraldsson
\vspace*{\stretch{2}} \normalsize %
\begin{flushleft}
IT University of Copenhagen \\
MDMI, S2015\\
Anders Hartvig Hartzen\\
Hajira Jabeen\\
Héctor Pérez Martínez\\
Sebastian Risi\\
Noor Shaker\\
\today \end{flushleft}
\end{titlepage}

\tableofcontents
\pagebreak
\section{Introduction}
\subsection{Data selection}
We have worked on a data set regarding the passengers on the titanic. The data structure is presented in table \ref{titanData} with a short name and the value type associated with the variable.
\begin{table}[h]
\begin{tabular}{|l|l|l|}
\hline
Variable name & Short name & Value Type\\
\hline
survival & Survived & binary\\
pclass & Passenger Class & numeric\\
name & Name & string\\
sex & Sex & binary string\\
age & Age & numeric\\
sibsp & Number of Siblings/Spouses Aboard & numeric\\
parch & Number of Parents/Children Aboard & numeric\\
ticket & Ticket Number & string\\
fare & Passenger Fare & numeric\\
cabin & Cabin & string\\
embarked & Port of Embarkation & attribute table\\
\hline
\end{tabular}
\caption{Titanic data set variables with short description and classification.}
\label{titanData}
\end{table}

The type binary and binary string, means there is only two options, the first of the types is based on numeric binary data, whilst the second is based on string data. The value type attribute table covers three options. The type designation in the table are based on how they are treated in the data mining code. This means that whilst some of the other variables might be a three option choice, it will not be treated as such. The reason for this division is mostly due to the nature of string variables and numeric variables.

\subsubsection{Data description}
The variables in the table are specified as follows. 

Survival is simply a 1 for survived and 0 for not. 

Passenger Class is a numeric value which can take on either 1, 2, or 3. The lower the value, the higher the class. The class is a proxy for the passengers socio-economic status.

Name is simply the name of the passenger, starting with surname.

Sex is the sex of the passenger.

Age is the age in years of the passenger. If the age is less than one it is fractional, and if it is estimated the age is in the form xx.5, meaning it has a decimal value as well.

Number of Siblings/Spouses Aboard, is the number of relatives, yet some of the relations are ignored. The meaning of sibling and spouse is summarized below.

Number of Parents/Children Aboard, is similar to the Number of Siblings/Spouses Aboard. The meaning is as well summarized below.
\begin{itemize}
	\item[\textbf{Sibling:}] Brother, Sister, Stepbrother, or Stepsister of Passenger Aboard Titanic
	\item[\textbf{Spouse:}] Husband or Wife of Passenger Aboard Titanic (Mistresses and Fiances Ignored)
\item[\textbf{Parent:}] Mother or Father of Passenger Aboard Titanic
\item[\textbf{Child:}] Son, Daughter, Stepson, or Stepdaughter of Passenger Aboard Titanic
\end{itemize}

Other family relatives excluded from this study include cousins, nephews/nieces, aunts/uncles, and in-laws.  Some children travelled only with a nanny, therefore parch=0 for them.  As well, some travelled with very close friends or neighbours in a village, however, the definitions do not support such relations.

Ticket Number is the number printed on the ticket. It is basically an identification string with characters and numbers. The tickets have no real pattern.

Passenger Fare is how much the passenger paid for its ticket. The price probably is listed in dollars.

Cabin is the cabin number. Again this is an identification of the rooms, yet there is not really any good patterns within the identification string.

Port of Embarkation is the place of embarking for the passenger. The port is abbreviated as follows. C is for Cherbourg,	Q is for Queenstown, S is for Southampton.

\subsection{Research question}
Our primary question which we want answered was:\\
\textit{"Which attributes contributes mostly to the survival rate of a passenger on the Titanic}\\
Here we wished to figure out what set of parameters would ensure the highest rate of survival on the Titanic. We would use classification through a classification tree to figure out which set gives the highest percentage of survival.

The secondary question arose when looking at clusters of the data.\\
\textit{"Which societal data can be found in clusters of the titanic data"}\\
Looking at the data, we decide to try clustering to see if there was an emergent pattern. It would be interesting to see if the were a relation between wealth and number of children and the like. 
\subsection{Tools for data mining}
We choose to use the free tool called Orange\cite{orange}. This tool allowed to quickly manipulate the data, such that we could extract the interesting elements. 
In the tool you manipulate the data by clicking and drawing connections between data elements and processing methods. Before explaining further, we have included the map of the process used for our data set in the orange framework, see figure \ref{OrangeMap}.

\begin{figure}[h]
	\centering
	\includegraphics[scale=0.35]{orangeMap}
	\caption{The map of the methods and processes used on the data.}
	\label{OrangeMap}
\end{figure}


\section{Data mining}
\subsection{Preprocessing}
\subsection{Classification tree}
Decided to use classification tree to try to accurately predict the survival of passengers from common known values such as, sex, age, number of siblings/spouses on board, number of parents/children on board, how much they paid for their ticket and where they embarked from.
\subsubsection{Cross Validation}
10 fold Cross validation
\subsection{K-means Clustering}
K-Means
We decided to use K-Means Clustering in order to increase the accuracy of our Classification Tree by adding the clusters to the attribute of the original data.

We started by normalizing the numerical values, 
We wanted to use that to increase the accuracy of the Classification tree

Numerical data was normalized before clustering

Did not have the desired effect due to data being clustered around the same points as the classification tree

Instead we saw interesting facts about the clustering.

There are 3 clusters total, 2 for males, 1 for females.

cluster 1 is all males, mostly from lower class and most of the embarked from Southampton(S) and they mostly died.

Cluster 2 is all female, no specific class, most of them embarked from Southampton(S) but most of them lived.

Cluster 3 is mostly male, mostly upper class, most of them embarked from Cherbourg and majority survived.

From this we can see that there not many upper class people living in Southampton and not a lot of people embarked from Queenstown(Q). Surviving as a male on the titanic meant that you had a much better chance if you were upper class and boarded from Cherbourg.

\begin{table}[h]
\begin{tabular}{|l|l|l|l|l|l|l|}
\hline
Sex & Embarked & Class & Age & Sibling/Spouse & Parent/Children & Fare\\
\hline
Male & S & 2.428 & 30.28 & 0 & 0 & 21.5697\\
Female & S & 2.258 & 26.32 & 1 & 1 & 35.2690\\
Male & C & 1.552 & 34.14 & 0 & 0 & 66.8577\\
\hline
\end{tabular}
\caption{Cluster Centroids after clustering.}
\label{clusterCentroids}
\end{table}

\begin{figure}[h]
	\centering
	\begin{center}
		\includegraphics[scale=0.30]{ClusterDistribution/Cluster1/Sex}
		\includegraphics[scale=0.30]{ClusterDistribution/Cluster1/PClass}\\
		\vspace{1 mm}
		\includegraphics[scale=0.30]{ClusterDistribution/Cluster1/Embarked}
		\includegraphics[scale=0.30]{ClusterDistribution/Cluster1/Fare}
	\end{center}
	\caption{Cluster 1}
	\label{ClusterOne}
\end{figure}

\begin{figure}[h]
	\centering
	\begin{center}
		\includegraphics[scale=0.30]{ClusterDistribution/Cluster2/Sex}
		\includegraphics[scale=0.30]{ClusterDistribution/Cluster2/PClass}\\
		\vspace{1 mm}
		\includegraphics[scale=0.30]{ClusterDistribution/Cluster2/Embarked}
		\includegraphics[scale=0.30]{ClusterDistribution/Cluster2/Fare}
	\end{center}
	\caption{Cluster 2}
	\label{ClusterTwo}
\end{figure}

\begin{figure}[h]
	\centering
	\begin{center}
		\includegraphics[scale=0.30]{ClusterDistribution/Cluster3/Sex}
		\includegraphics[scale=0.30]{ClusterDistribution/Cluster3/PClass}\\
		\vspace{1 mm}
		\includegraphics[scale=0.30]{ClusterDistribution/Cluster3/Embarked}
		\includegraphics[scale=0.30]{ClusterDistribution/Cluster3/Fare}
	\end{center}
	\caption{Cluster 3}
	\label{ClusterThree}
\end{figure}


\subsubsection{Data validation}
10 fold cross validation

\section{Conclusion}
\subsection{Societal impact}
Time travellers could use this information to go on the titanic to have the adventure of a lifetime trying to use statistics to survive.

The classification tree could be used to predict the survival of people whose whereabouts were unknown.




\appendix
\begin{thebibliography}{9}

\bibitem{orange}
  \emph{Orange Data Mining},
  http://orange.biolab.si/,
  13-05-15
\end{thebibliography}

\end{document}


